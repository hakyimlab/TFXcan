% Options for packages loaded elsewhere
\PassOptionsToPackage{unicode}{hyperref}
\PassOptionsToPackage{hyphens}{url}
\PassOptionsToPackage{dvipsnames,svgnames,x11names}{xcolor}
%
\documentclass[
  letterpaper,
  DIV=11,
  numbers=noendperiod]{scrartcl}

\usepackage{amsmath,amssymb}
\usepackage{lmodern}
\usepackage{iftex}
\ifPDFTeX
  \usepackage[T1]{fontenc}
  \usepackage[utf8]{inputenc}
  \usepackage{textcomp} % provide euro and other symbols
\else % if luatex or xetex
  \usepackage{unicode-math}
  \defaultfontfeatures{Scale=MatchLowercase}
  \defaultfontfeatures[\rmfamily]{Ligatures=TeX,Scale=1}
\fi
% Use upquote if available, for straight quotes in verbatim environments
\IfFileExists{upquote.sty}{\usepackage{upquote}}{}
\IfFileExists{microtype.sty}{% use microtype if available
  \usepackage[]{microtype}
  \UseMicrotypeSet[protrusion]{basicmath} % disable protrusion for tt fonts
}{}
\makeatletter
\@ifundefined{KOMAClassName}{% if non-KOMA class
  \IfFileExists{parskip.sty}{%
    \usepackage{parskip}
  }{% else
    \setlength{\parindent}{0pt}
    \setlength{\parskip}{6pt plus 2pt minus 1pt}}
}{% if KOMA class
  \KOMAoptions{parskip=half}}
\makeatother
\usepackage{xcolor}
\setlength{\emergencystretch}{3em} % prevent overfull lines
\setcounter{secnumdepth}{5}
% Make \paragraph and \subparagraph free-standing
\ifx\paragraph\undefined\else
  \let\oldparagraph\paragraph
  \renewcommand{\paragraph}[1]{\oldparagraph{#1}\mbox{}}
\fi
\ifx\subparagraph\undefined\else
  \let\oldsubparagraph\subparagraph
  \renewcommand{\subparagraph}[1]{\oldsubparagraph{#1}\mbox{}}
\fi

\usepackage{color}
\usepackage{fancyvrb}
\newcommand{\VerbBar}{|}
\newcommand{\VERB}{\Verb[commandchars=\\\{\}]}
\DefineVerbatimEnvironment{Highlighting}{Verbatim}{commandchars=\\\{\}}
% Add ',fontsize=\small' for more characters per line
\usepackage{framed}
\definecolor{shadecolor}{RGB}{241,243,245}
\newenvironment{Shaded}{\begin{snugshade}}{\end{snugshade}}
\newcommand{\AlertTok}[1]{\textcolor[rgb]{0.68,0.00,0.00}{#1}}
\newcommand{\AnnotationTok}[1]{\textcolor[rgb]{0.37,0.37,0.37}{#1}}
\newcommand{\AttributeTok}[1]{\textcolor[rgb]{0.40,0.45,0.13}{#1}}
\newcommand{\BaseNTok}[1]{\textcolor[rgb]{0.68,0.00,0.00}{#1}}
\newcommand{\BuiltInTok}[1]{\textcolor[rgb]{0.00,0.23,0.31}{#1}}
\newcommand{\CharTok}[1]{\textcolor[rgb]{0.13,0.47,0.30}{#1}}
\newcommand{\CommentTok}[1]{\textcolor[rgb]{0.37,0.37,0.37}{#1}}
\newcommand{\CommentVarTok}[1]{\textcolor[rgb]{0.37,0.37,0.37}{\textit{#1}}}
\newcommand{\ConstantTok}[1]{\textcolor[rgb]{0.56,0.35,0.01}{#1}}
\newcommand{\ControlFlowTok}[1]{\textcolor[rgb]{0.00,0.23,0.31}{#1}}
\newcommand{\DataTypeTok}[1]{\textcolor[rgb]{0.68,0.00,0.00}{#1}}
\newcommand{\DecValTok}[1]{\textcolor[rgb]{0.68,0.00,0.00}{#1}}
\newcommand{\DocumentationTok}[1]{\textcolor[rgb]{0.37,0.37,0.37}{\textit{#1}}}
\newcommand{\ErrorTok}[1]{\textcolor[rgb]{0.68,0.00,0.00}{#1}}
\newcommand{\ExtensionTok}[1]{\textcolor[rgb]{0.00,0.23,0.31}{#1}}
\newcommand{\FloatTok}[1]{\textcolor[rgb]{0.68,0.00,0.00}{#1}}
\newcommand{\FunctionTok}[1]{\textcolor[rgb]{0.28,0.35,0.67}{#1}}
\newcommand{\ImportTok}[1]{\textcolor[rgb]{0.00,0.46,0.62}{#1}}
\newcommand{\InformationTok}[1]{\textcolor[rgb]{0.37,0.37,0.37}{#1}}
\newcommand{\KeywordTok}[1]{\textcolor[rgb]{0.00,0.23,0.31}{#1}}
\newcommand{\NormalTok}[1]{\textcolor[rgb]{0.00,0.23,0.31}{#1}}
\newcommand{\OperatorTok}[1]{\textcolor[rgb]{0.37,0.37,0.37}{#1}}
\newcommand{\OtherTok}[1]{\textcolor[rgb]{0.00,0.23,0.31}{#1}}
\newcommand{\PreprocessorTok}[1]{\textcolor[rgb]{0.68,0.00,0.00}{#1}}
\newcommand{\RegionMarkerTok}[1]{\textcolor[rgb]{0.00,0.23,0.31}{#1}}
\newcommand{\SpecialCharTok}[1]{\textcolor[rgb]{0.37,0.37,0.37}{#1}}
\newcommand{\SpecialStringTok}[1]{\textcolor[rgb]{0.13,0.47,0.30}{#1}}
\newcommand{\StringTok}[1]{\textcolor[rgb]{0.13,0.47,0.30}{#1}}
\newcommand{\VariableTok}[1]{\textcolor[rgb]{0.07,0.07,0.07}{#1}}
\newcommand{\VerbatimStringTok}[1]{\textcolor[rgb]{0.13,0.47,0.30}{#1}}
\newcommand{\WarningTok}[1]{\textcolor[rgb]{0.37,0.37,0.37}{\textit{#1}}}

\providecommand{\tightlist}{%
  \setlength{\itemsep}{0pt}\setlength{\parskip}{0pt}}\usepackage{longtable,booktabs,array}
\usepackage{calc} % for calculating minipage widths
% Correct order of tables after \paragraph or \subparagraph
\usepackage{etoolbox}
\makeatletter
\patchcmd\longtable{\par}{\if@noskipsec\mbox{}\fi\par}{}{}
\makeatother
% Allow footnotes in longtable head/foot
\IfFileExists{footnotehyper.sty}{\usepackage{footnotehyper}}{\usepackage{footnote}}
\makesavenoteenv{longtable}
\usepackage{graphicx}
\makeatletter
\def\maxwidth{\ifdim\Gin@nat@width>\linewidth\linewidth\else\Gin@nat@width\fi}
\def\maxheight{\ifdim\Gin@nat@height>\textheight\textheight\else\Gin@nat@height\fi}
\makeatother
% Scale images if necessary, so that they will not overflow the page
% margins by default, and it is still possible to overwrite the defaults
% using explicit options in \includegraphics[width, height, ...]{}
\setkeys{Gin}{width=\maxwidth,height=\maxheight,keepaspectratio}
% Set default figure placement to htbp
\makeatletter
\def\fps@figure{htbp}
\makeatother

\KOMAoption{captions}{tableheading}
\makeatletter
\makeatother
\makeatletter
\makeatother
\makeatletter
\@ifpackageloaded{caption}{}{\usepackage{caption}}
\AtBeginDocument{%
\ifdefined\contentsname
  \renewcommand*\contentsname{Table of contents}
\else
  \newcommand\contentsname{Table of contents}
\fi
\ifdefined\listfigurename
  \renewcommand*\listfigurename{List of Figures}
\else
  \newcommand\listfigurename{List of Figures}
\fi
\ifdefined\listtablename
  \renewcommand*\listtablename{List of Tables}
\else
  \newcommand\listtablename{List of Tables}
\fi
\ifdefined\figurename
  \renewcommand*\figurename{Figure}
\else
  \newcommand\figurename{Figure}
\fi
\ifdefined\tablename
  \renewcommand*\tablename{Table}
\else
  \newcommand\tablename{Table}
\fi
}
\@ifpackageloaded{float}{}{\usepackage{float}}
\floatstyle{ruled}
\@ifundefined{c@chapter}{\newfloat{codelisting}{h}{lop}}{\newfloat{codelisting}{h}{lop}[chapter]}
\floatname{codelisting}{Listing}
\newcommand*\listoflistings{\listof{codelisting}{List of Listings}}
\makeatother
\makeatletter
\@ifpackageloaded{caption}{}{\usepackage{caption}}
\@ifpackageloaded{subcaption}{}{\usepackage{subcaption}}
\makeatother
\makeatletter
\@ifpackageloaded{tcolorbox}{}{\usepackage[many]{tcolorbox}}
\makeatother
\makeatletter
\@ifundefined{shadecolor}{\definecolor{shadecolor}{rgb}{.97, .97, .97}}
\makeatother
\makeatletter
\makeatother
\ifLuaTeX
  \usepackage{selnolig}  % disable illegal ligatures
\fi
\IfFileExists{bookmark.sty}{\usepackage{bookmark}}{\usepackage{hyperref}}
\IfFileExists{xurl.sty}{\usepackage{xurl}}{} % add URL line breaks if available
\urlstyle{same} % disable monospaced font for URLs
\hypersetup{
  pdftitle={TF binding analysis 2 \textbar{} Model comparisons},
  pdfauthor={Temi},
  colorlinks=true,
  linkcolor={blue},
  filecolor={Maroon},
  citecolor={Blue},
  urlcolor={Blue},
  pdfcreator={LaTeX via pandoc}}

\title{TF binding analysis 2 \textbar{} Model comparisons}
\author{Temi}
\date{}

\begin{document}
\maketitle
\ifdefined\Shaded\renewenvironment{Shaded}{\begin{tcolorbox}[borderline west={3pt}{0pt}{shadecolor}, breakable, sharp corners, boxrule=0pt, frame hidden, interior hidden, enhanced]}{\end{tcolorbox}}\fi

\renewcommand*\contentsname{Table of contents}
{
\hypersetup{linkcolor=}
\setcounter{tocdepth}{3}
\tableofcontents
}
\begin{Shaded}
\begin{Highlighting}[numbers=left,,]
\NormalTok{knitr}\SpecialCharTok{::}\NormalTok{opts\_chunk}\SpecialCharTok{$}\FunctionTok{set}\NormalTok{(}\AttributeTok{fig.width=}\DecValTok{12}\NormalTok{, }\AttributeTok{fig.height=}\DecValTok{8}\NormalTok{, }\AttributeTok{cache=}\NormalTok{T)}
\end{Highlighting}
\end{Shaded}

\begin{Shaded}
\begin{Highlighting}[numbers=left,,]
\FunctionTok{rm}\NormalTok{(}\AttributeTok{list=}\FunctionTok{ls}\NormalTok{())}

\FunctionTok{setwd}\NormalTok{(}\StringTok{\textquotesingle{}/projects/covid{-}ct/imlab/users/temi/projects/TFXcan/scripts/\textquotesingle{}}\NormalTok{)}

\FunctionTok{library}\NormalTok{(glue)}
\FunctionTok{library}\NormalTok{(R.utils)}
\FunctionTok{library}\NormalTok{(data.table)}
\FunctionTok{library}\NormalTok{(glmnet)}
\FunctionTok{library}\NormalTok{(doMC)}
\FunctionTok{library}\NormalTok{(ROCR)}
\FunctionTok{library}\NormalTok{(Matrix)}
\FunctionTok{library}\NormalTok{(reshape2)}
\FunctionTok{library}\NormalTok{(tidyverse)}
\FunctionTok{library}\NormalTok{(foreach)}
\FunctionTok{library}\NormalTok{(doParallel)}
\FunctionTok{library}\NormalTok{(tidymodels)}
\FunctionTok{library}\NormalTok{(broom)}
\end{Highlighting}
\end{Shaded}

\begin{Shaded}
\begin{Highlighting}[numbers=left,,]
\NormalTok{plots\_dir }\OtherTok{\textless{}{-}} \StringTok{\textquotesingle{}../plots\textquotesingle{}}
\NormalTok{models\_dir }\OtherTok{\textless{}{-}} \StringTok{\textquotesingle{}../models\textquotesingle{}}
\end{Highlighting}
\end{Shaded}

\begin{Shaded}
\begin{Highlighting}[numbers=left,,]
\NormalTok{TF }\OtherTok{\textless{}{-}} \StringTok{\textquotesingle{}GATA3\textquotesingle{}}
\NormalTok{cistrome\_dir }\OtherTok{\textless{}{-}} \StringTok{\textquotesingle{}/projects/covid{-}ct/imlab/data/cistrome/compressed\textquotesingle{}}
\NormalTok{hf\_info }\OtherTok{\textless{}{-}}\NormalTok{ data.table}\SpecialCharTok{::}\FunctionTok{fread}\NormalTok{(}\FunctionTok{glue}\NormalTok{(}\StringTok{\textquotesingle{}\{cistrome\_dir\}/human\_factor\_full\_QC.txt\textquotesingle{}}\NormalTok{))}
\NormalTok{TF\_DCID }\OtherTok{\textless{}{-}}\NormalTok{ hf\_info[hf\_info}\SpecialCharTok{$}\NormalTok{Factor }\SpecialCharTok{==}\NormalTok{ TF, ]}
\FunctionTok{head}\NormalTok{(TF\_DCID)}
\end{Highlighting}
\end{Shaded}

\begin{verbatim}
    DCid      Species     GSMID Factor Cell_line       Cell_type Tissue_type
1:  2324 Homo sapiens GSM720423  GATA3     MCF-7      Epithelium      Breast
2:  2325 Homo sapiens GSM720422  GATA3     MCF-7      Epithelium      Breast
3:  9195 Homo sapiens GSM957608  GATA3 RPMI-8402 T cell leukemia       Blood
4:  9196 Homo sapiens GSM957609  GATA3 RPMI-8402 T cell leukemia       Blood
5: 33129 Homo sapiens GSM986070  GATA3     MCF-7      Epithelium      Breast
6: 33136 Homo sapiens GSM986075  GATA3     MCF-7      Epithelium      Breast
   FastQC UniquelyMappedRatio   PBC PeaksFoldChangeAbove10       FRiP
1:     30              0.7981 0.980                   5420 0.03490225
2:     30              0.8010 0.989                  12454 0.08374850
3:     39              0.7523 0.244                   1432 0.08693400
4:     39              0.7604 0.808                   3459 0.04521850
5:     39              0.8001 0.990                   8318 0.04329025
6:     39              0.7460 0.971                    358 0.00357925
   PeaksUnionDHSRatio
1:          0.9750000
2:          0.9810000
3:          0.8848000
4:          0.9524000
5:          0.9618000
6:          0.8949343
\end{verbatim}

https://stats.stackexchange.com/questions/138569/why-is-lambda-within-one-standard-error-from-the-minimum-is-a-recommended-valu

\newpage

For plotting, I need to load in the enformer annotations

\begin{Shaded}
\begin{Highlighting}[numbers=left,,]
\NormalTok{enformer\_annotations }\OtherTok{\textless{}{-}}\NormalTok{ data.table}\SpecialCharTok{::}\FunctionTok{fread}\NormalTok{(}\StringTok{\textquotesingle{}../info{-}files/enformer\_tracks\_annotated{-}resaved.txt\textquotesingle{}}\NormalTok{)}
\NormalTok{enformer\_annotations }\OtherTok{\textless{}{-}}\NormalTok{ enformer\_annotations[}\SpecialCharTok{!}\FunctionTok{is.na}\NormalTok{(enformer\_annotations}\SpecialCharTok{$}\NormalTok{assay), ]}
\end{Highlighting}
\end{Shaded}

\begin{Shaded}
\begin{Highlighting}[numbers=left,,]
\CommentTok{\# using an individual for now \textgreater{}\textgreater{} will do this across all 3 individuals later}
\NormalTok{ind }\OtherTok{\textless{}{-}} \StringTok{\textquotesingle{}HG00479\textquotesingle{}}
\end{Highlighting}
\end{Shaded}

There are 5 major categories

\begin{Shaded}
\begin{Highlighting}[numbers=left,,]
\NormalTok{assays }\OtherTok{\textless{}{-}}\NormalTok{ enformer\_annotations}\SpecialCharTok{$}\NormalTok{assay }\SpecialCharTok{|\textgreater{}} \FunctionTok{unique}\NormalTok{()}
\NormalTok{assays}
\end{Highlighting}
\end{Shaded}

\begin{verbatim}
[1] "DNase-seq"        "ATAC-seq"         "TF ChIP-seq"      "Histone ChIP-seq"
[5] "CAGE experiment" 
\end{verbatim}

\begin{Shaded}
\begin{Highlighting}[numbers=left,,]
\NormalTok{z\_list }\OtherTok{\textless{}{-}} \FunctionTok{list}\NormalTok{()}
\ControlFlowTok{for}\NormalTok{(i }\ControlFlowTok{in} \DecValTok{1}\SpecialCharTok{:}\FunctionTok{length}\NormalTok{(assays))\{}
\NormalTok{    z }\OtherTok{\textless{}{-}} \FunctionTok{combn}\NormalTok{(assays, i)}
\NormalTok{    z\_list[[i]] }\OtherTok{\textless{}{-}} \FunctionTok{t}\NormalTok{(z) }\SpecialCharTok{|\textgreater{}} \FunctionTok{as.data.frame}\NormalTok{()}
\NormalTok{\}}

\NormalTok{assays\_dt }\OtherTok{\textless{}{-}}\NormalTok{ plyr}\SpecialCharTok{::}\FunctionTok{rbind.fill}\NormalTok{(z\_list)}
\end{Highlighting}
\end{Shaded}

\hypertarget{helper-functions}{%
\subsubsection{helper functions}\label{helper-functions}}

\begin{Shaded}
\begin{Highlighting}[numbers=left,,]
\NormalTok{collate\_coefficients }\OtherTok{\textless{}{-}} \ControlFlowTok{function}\NormalTok{(fit)\{}
\NormalTok{    min\_error\_index }\OtherTok{\textless{}{-}}\NormalTok{ fit}\SpecialCharTok{$}\NormalTok{index[}\StringTok{\textquotesingle{}min\textquotesingle{}}\NormalTok{, ]}
\NormalTok{    one\_sd\_index }\OtherTok{\textless{}{-}}\NormalTok{ fit}\SpecialCharTok{$}\NormalTok{index[}\StringTok{\textquotesingle{}1se\textquotesingle{}}\NormalTok{, ]}

\NormalTok{    dimensions }\OtherTok{\textless{}{-}}\NormalTok{ fit}\SpecialCharTok{$}\NormalTok{glmnet.fit}\SpecialCharTok{$}\NormalTok{beta}\SpecialCharTok{@}\NormalTok{Dim}
\NormalTok{    coef\_mat }\OtherTok{\textless{}{-}} \FunctionTok{as.data.frame}\NormalTok{(}\FunctionTok{summary}\NormalTok{(fit}\SpecialCharTok{$}\NormalTok{glmnet.fit}\SpecialCharTok{$}\NormalTok{beta))}

\NormalTok{    temp\_mat }\OtherTok{\textless{}{-}} \FunctionTok{matrix}\NormalTok{(}\AttributeTok{data=}\ConstantTok{NA}\NormalTok{, }\AttributeTok{nrow=}\NormalTok{dimensions[}\DecValTok{1}\NormalTok{], }\AttributeTok{ncol=}\NormalTok{dimensions[}\DecValTok{2}\NormalTok{])}
    \CommentTok{\#print(dim(temp\_mat))}
    \ControlFlowTok{for}\NormalTok{(i }\ControlFlowTok{in} \DecValTok{1}\SpecialCharTok{:}\FunctionTok{nrow}\NormalTok{(coef\_mat))\{}
\NormalTok{        temp\_mat[coef\_mat[i, }\StringTok{\textquotesingle{}i\textquotesingle{}}\NormalTok{], coef\_mat[i, }\StringTok{\textquotesingle{}j\textquotesingle{}}\NormalTok{]] }\OtherTok{\textless{}{-}}\NormalTok{ coef\_mat[i, }\StringTok{\textquotesingle{}x\textquotesingle{}}\NormalTok{]}
\NormalTok{    \}}

\NormalTok{    temp\_mat[}\FunctionTok{is.na}\NormalTok{(temp\_mat)] }\OtherTok{\textless{}{-}} \DecValTok{0}

\NormalTok{    fit\_beta }\OtherTok{\textless{}{-}} \FunctionTok{cbind}\NormalTok{(temp\_mat[, min\_error\_index], temp\_mat[, one\_sd\_index])}
    \CommentTok{\# what features were used?}
\NormalTok{    feature\_data }\OtherTok{\textless{}{-}}\NormalTok{ enformer\_annotations[enformer\_annotations}\SpecialCharTok{$}\NormalTok{feature\_names }\SpecialCharTok{\%in\%}\NormalTok{ (fit}\SpecialCharTok{$}\NormalTok{glmnet.fit}\SpecialCharTok{$}\NormalTok{beta }\SpecialCharTok{|\textgreater{}} \FunctionTok{rownames}\NormalTok{()), }\FunctionTok{c}\NormalTok{(}\StringTok{\textquotesingle{}assay\textquotesingle{}}\NormalTok{, }\StringTok{\textquotesingle{}feature\_names\textquotesingle{}}\NormalTok{)]}

\NormalTok{    fit\_beta }\OtherTok{\textless{}{-}} \FunctionTok{as.data.frame}\NormalTok{(}\FunctionTok{cbind}\NormalTok{(fit\_beta, feature\_data))}

    \FunctionTok{return}\NormalTok{(fit\_beta)}
\NormalTok{\}}

\NormalTok{test\_models }\OtherTok{\textless{}{-}} \ControlFlowTok{function}\NormalTok{(model, X\_test\_set, y\_test\_set)\{}

\NormalTok{    features }\OtherTok{\textless{}{-}}\NormalTok{ model}\SpecialCharTok{$}\NormalTok{glmnet.fit}\SpecialCharTok{$}\NormalTok{beta }\SpecialCharTok{|\textgreater{}} \FunctionTok{rownames}\NormalTok{()}
\NormalTok{    X\_test\_set }\OtherTok{\textless{}{-}}\NormalTok{ X\_test\_set[, features]}
    \FunctionTok{assess.glmnet}\NormalTok{(model, }\AttributeTok{newx =}\NormalTok{ X\_test\_set, }\AttributeTok{newy =}\NormalTok{ y\_test\_set) }\SpecialCharTok{|\textgreater{}} \FunctionTok{unlist}\NormalTok{()}
\NormalTok{\}}
\end{Highlighting}
\end{Shaded}

\hypertarget{load-the-models}{%
\subsection{Load the models}\label{load-the-models}}

\begin{Shaded}
\begin{Highlighting}[numbers=left,,]
\NormalTok{aggModels\_names }\OtherTok{\textless{}{-}} \FunctionTok{c}\NormalTok{(}\StringTok{\textquotesingle{}aggByCenter\textquotesingle{}}\NormalTok{, }\StringTok{\textquotesingle{}aggByMean\textquotesingle{}}\NormalTok{, }\StringTok{\textquotesingle{}aggByMeanUpstream\textquotesingle{}}\NormalTok{, }\StringTok{\textquotesingle{}aggByMeanDownstream\textquotesingle{}}\NormalTok{, }\StringTok{\textquotesingle{}aggByMeanUpstreamDownstream\textquotesingle{}}\NormalTok{)}

\NormalTok{aggModels }\OtherTok{\textless{}{-}} \FunctionTok{lapply}\NormalTok{(aggModels\_names, }\ControlFlowTok{function}\NormalTok{(each\_agg)\{}
    \FunctionTok{readRDS}\NormalTok{(}\FunctionTok{glue}\NormalTok{(}\StringTok{\textquotesingle{}\{models\_dir\}/enet\_models\_\{each\_agg\}.RData\textquotesingle{}}\NormalTok{))}
\NormalTok{\})}

\FunctionTok{names}\NormalTok{(aggModels) }\OtherTok{\textless{}{-}}\NormalTok{ aggModels\_names}
\end{Highlighting}
\end{Shaded}

I need to curate predictions across the models (without a test set yet)

\begin{Shaded}
\begin{Highlighting}[numbers=left,,]
\NormalTok{cv\_metrics }\OtherTok{\textless{}{-}} \ControlFlowTok{function}\NormalTok{(fit, }\AttributeTok{lambda =} \StringTok{\textquotesingle{}lambda.1se\textquotesingle{}}\NormalTok{)\{}
\NormalTok{    whlm }\OtherTok{\textless{}{-}} \FunctionTok{which}\NormalTok{(fit}\SpecialCharTok{$}\NormalTok{lambda }\SpecialCharTok{==}\NormalTok{ fit[[lambda]])}
    \CommentTok{\# mse}
\NormalTok{    mse }\OtherTok{\textless{}{-}}\NormalTok{ fit}\SpecialCharTok{$}\NormalTok{cvm[whlm]}
\NormalTok{    out }\OtherTok{\textless{}{-}} \FunctionTok{with}\NormalTok{(fit}\SpecialCharTok{$}\NormalTok{glmnet.fit,}
\NormalTok{        \{}
\NormalTok{            tLL }\OtherTok{\textless{}{-}}\NormalTok{ nulldev }\SpecialCharTok{{-}}\NormalTok{ nulldev }\SpecialCharTok{*}\NormalTok{ (}\DecValTok{1} \SpecialCharTok{{-}}\NormalTok{ dev.ratio)[whlm]}
\NormalTok{            k }\OtherTok{\textless{}{-}}\NormalTok{ df[whlm]}
\NormalTok{            n }\OtherTok{\textless{}{-}}\NormalTok{ nobs}
            \FunctionTok{list}\NormalTok{(}\StringTok{\textquotesingle{}AICc\textquotesingle{}} \OtherTok{=} \SpecialCharTok{{-}}\NormalTok{ tLL }\SpecialCharTok{+} \DecValTok{2} \SpecialCharTok{*}\NormalTok{ k }\SpecialCharTok{+} \DecValTok{2} \SpecialCharTok{*}\NormalTok{ k }\SpecialCharTok{*}\NormalTok{ (k }\SpecialCharTok{+} \DecValTok{1}\NormalTok{) }\SpecialCharTok{/}\NormalTok{ (n }\SpecialCharTok{{-}}\NormalTok{ k }\SpecialCharTok{{-}} \DecValTok{1}\NormalTok{),}
                        \StringTok{\textquotesingle{}BIC\textquotesingle{}} \OtherTok{=} \FunctionTok{log}\NormalTok{(n) }\SpecialCharTok{*}\NormalTok{ k }\SpecialCharTok{{-}}\NormalTok{ tLL)}
\NormalTok{        \})}

\NormalTok{    out}\SpecialCharTok{$}\NormalTok{mse }\OtherTok{\textless{}{-}}\NormalTok{ mse}
    \FunctionTok{return}\NormalTok{(}\FunctionTok{do.call}\NormalTok{(cbind, out) }\SpecialCharTok{|\textgreater{}} \FunctionTok{as.data.frame}\NormalTok{())}
\NormalTok{\}}
\end{Highlighting}
\end{Shaded}

\begin{Shaded}
\begin{Highlighting}[numbers=left,,]
\NormalTok{aggModels}\SpecialCharTok{$}\NormalTok{aggByCenter[[}\DecValTok{1}\NormalTok{]] }\SpecialCharTok{|\textgreater{}} \FunctionTok{cv\_metrics}\NormalTok{()}
\end{Highlighting}
\end{Shaded}

\begin{verbatim}
       AICc      BIC       mse
1 -628.7396 709.6987 0.8497953
\end{verbatim}

\begin{Shaded}
\begin{Highlighting}[numbers=left,,]
\NormalTok{aggModels\_metrics }\OtherTok{\textless{}{-}} \FunctionTok{lapply}\NormalTok{(aggModels, }\ControlFlowTok{function}\NormalTok{(each\_model\_list)\{}
\NormalTok{    out }\OtherTok{\textless{}{-}} \FunctionTok{do.call}\NormalTok{(rbind, }\FunctionTok{lapply}\NormalTok{(each\_model\_list, cv\_metrics))}
\NormalTok{\})}
\FunctionTok{names}\NormalTok{(aggModels\_metrics) }\OtherTok{\textless{}{-}}\NormalTok{ aggModels\_names}
\end{Highlighting}
\end{Shaded}

\hypertarget{predict-on-the-test-set}{%
\subsubsection{predict on the test set}\label{predict-on-the-test-set}}

\begin{Shaded}
\begin{Highlighting}[numbers=left,,]
\NormalTok{aggModels\_predict\_test }\OtherTok{\textless{}{-}} \FunctionTok{lapply}\NormalTok{(}\FunctionTok{seq\_along}\NormalTok{(aggModels), }\ControlFlowTok{function}\NormalTok{(i)\{}
\NormalTok{    test\_data }\OtherTok{\textless{}{-}} \FunctionTok{read.csv}\NormalTok{(}\FunctionTok{glue}\NormalTok{(}\StringTok{\textquotesingle{}/projects/covid{-}ct/imlab/users/temi/projects/TFXcan/output/train{-}test{-}data/test\_\{names(aggModels[i])\}\_\{ind\}.csv\textquotesingle{}}\NormalTok{))}

\NormalTok{    X\_test }\OtherTok{\textless{}{-}} \FunctionTok{as.matrix}\NormalTok{(test\_data[, }\SpecialCharTok{{-}}\FunctionTok{c}\NormalTok{(}\DecValTok{1}\NormalTok{)])}
\NormalTok{    y\_test }\OtherTok{\textless{}{-}} \FunctionTok{as.matrix}\NormalTok{(test\_data[, }\DecValTok{1}\NormalTok{])}

    \FunctionTok{do.call}\NormalTok{(rbind, }\FunctionTok{lapply}\NormalTok{(aggModels[[i]], test\_models, X\_test, y\_test)) }\SpecialCharTok{|\textgreater{}} \FunctionTok{as.data.frame}\NormalTok{()}
\NormalTok{\})}
\end{Highlighting}
\end{Shaded}

\(y_{i} \sim \sum_{i=1}^{n}\sum_{j=1}^{m}x_{ij}\beta_{ij} + \epsilon\)

\(log(\frac{p_{i}}{(1-p_{i})}) = \beta_{0} + \sum_{i=1}^{n}\sum_{j=1}^{m}x_{ij}\beta_{ij}\)

\(p_{i} = \frac{1}{1-e^{-x_{i}}} = \frac{1}{1-e^{-(\beta_{0} + \sum_{i=1}^{n}\sum_{j=1}^{m}x_{ij}\beta_{ij})}}\)

where \(n\) is the number of predictors, in this case the number of
tracks or thereabouts from ENFORMER, \(m\) is the number of categories,
in this case any of the following:

\begin{enumerate}
\def\labelenumi{\arabic{enumi}.}
\tightlist
\item
  ATAC-seq
\item
  TF ChIP-seq
\item
  Histone ChIP-seq
\item
  CAGE experiments
\item
  DNase-seq,
\end{enumerate}

\(y\) is the presence or absence of TF-binding, in this case GATA3
binding, \(x_{ij}\) is the predictor (from ENFORMER) of sample \(i\)
belonging to category \(j\) and aggregated by any of the following:

\begin{enumerate}
\def\labelenumi{\arabic{enumi}.}
\tightlist
\item
  mean of the 8 bins upstream and downstream the motif bin
\item
  just the motif bin
\item
  mean of all the bins
\item
  mean of only the upstream bins
\item
  mean of only the downstream bins.
\end{enumerate}



\end{document}
